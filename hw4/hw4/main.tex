

\documentclass[table,draft ]{article}

\usepackage{graphicx}			% Use this package to include images
\usepackage{amsmath}			% A library of many standard math expressions
\usepackage[margin=0.75in, a4paper]{geometry}% Sets 1in margins. 
\usepackage{fancyhdr}			% Creates headers and footers
\usepackage{enumerate}          %These two package give custom labels to a list
\usepackage[shortlabels]{enumitem}
\usepackage{tikz}
\usepackage{float} % Add this in your preamble for using [H]
\usepackage{subcaption}
\usepackage{booktabs}
\usepackage{caption}
\usepackage[normalem]{ulem}
\useunder{\uline}{\ul}{}
\usepackage{circuitikz}
\usepackage{ragged2e} % Package for justification
\justifying

\graphicspath{{img/}}
\usetikzlibrary{shapes.geometric, arrows, positioning}

\def\mytitle{Homework 4}
\def\righthead{EEF205E}


\usepackage[T1]{fontenc}
\usepackage{lmodern}

\usepackage{listings}
\usepackage{xcolor}
\usepackage{colortbl}
\usepackage{svg}
\usepackage{ifthen}
\usepackage{tikz}
\usepackage{karnaugh-map}
\usepackage{hyperref}

\usetikzlibrary{calc}




\pgfkeys{
  /sevenseg/.is family, /sevenseg,
  slant/.estore in      = \sevensegSlant,     % vertical slant in degrees
  size/.estore in       = \sevensegSize,      % length of a segment
  shrink/.estore in     = \sevensegShrink,    % avoids overlapping of segments
  line width/.estore in = \sevensegLinewidth, % thickness of the segments
  line cap/.estore in   = \sevensegLinecap,   % end cap style rect, round, butt
  oncolor/.estore in    = \sevensegOncolor,   % color of an ON segment
  offcolor/.estore in   = \sevensegOffcolor,  % color of an OFF segment
}

\pgfkeys{
  /sevenseg,
  default/.style = {slant = 0, size = 1em, shrink = 0.1, 
                    line width = 0.15em, line cap = round, 
                    oncolor = red, offcolor = lightgray}
}

%===============================================
%                     a b c d e f g - segment values
% \sevenseg[options]{{1,1,1,1,1,1,0,}}
%
\newcommand{\sevenseg}[2][]% options values
{%
\pgfkeys{/sevenseg, default, #1}%
\def\sevensegarray{#2}%
  \begin{tikzpicture}%
    % first define the position of the 6 corner points
    \path (0,0) ++(0,0)                             coordinate (P1);
    \path (0,0) ++(\sevensegSize,0)                 coordinate (P2);
    \path (0,0) ++(90-\sevensegSlant:\sevensegSize) coordinate (P3);
    \path (P2)  ++(90-\sevensegSlant:\sevensegSize) coordinate (P4);
    \path (P3)  ++(90-\sevensegSlant:\sevensegSize) coordinate (P5);
    \path (P4)  ++(90-\sevensegSlant:\sevensegSize) coordinate (P6);
    % then step through the 1/0 values in the segment array
    \foreach \i in {0,...,6}%
    {
      \pgfmathparse{\sevensegarray[\i]}
      \ifthenelse{\equal{\pgfmathresult}{1}}%
        {\let\mycolor=\sevensegOncolor}%  segment is on
        {\let\mycolor=\sevensegOffcolor}% segment is off
      \tikzstyle{segstyle} = [draw=\mycolor, line width = \sevensegLinewidth,
                              line cap = \sevensegLinecap]
      %-----------------------
      \ifthenelse{\equal{\i}{0}}{\path[segstyle] 
        (${1-\sevensegShrink}*(P5)+\sevensegShrink*(P6)$) 
        -- ($\sevensegShrink*(P5)+{1-\sevensegShrink}*(P6)$);}{} % a
      \ifthenelse{\equal{\i}{1}}{\path[segstyle] 
        (${1-\sevensegShrink}*(P6)+\sevensegShrink*(P4)$) 
        -- ($\sevensegShrink*(P6)+{1-\sevensegShrink}*(P4)$);}{} % b
      \ifthenelse{\equal{\i}{2}}{\path[segstyle] 
        (${1-\sevensegShrink}*(P4)+\sevensegShrink*(P2)$) 
        -- ($\sevensegShrink*(P4)+{1-\sevensegShrink}*(P2)$);}{} % c
      \ifthenelse{\equal{\i}{3}}{\path[segstyle] 
        (${1-\sevensegShrink}*(P1)+\sevensegShrink*(P2)$) 
        -- ($\sevensegShrink*(P1)+{1-\sevensegShrink}*(P2)$);}{} % d
      \ifthenelse{\equal{\i}{4}}{\path[segstyle] 
        (${1-\sevensegShrink}*(P1)+\sevensegShrink*(P3)$) 
        -- ($\sevensegShrink*(P1)+{1-\sevensegShrink}*(P3)$);}{} % e
      \ifthenelse{\equal{\i}{5}}{\path[segstyle] 
        (${1-\sevensegShrink}*(P3)+\sevensegShrink*(P5)$) 
        -- ($\sevensegShrink*(P3)+{1-\sevensegShrink}*(P5)$);}{} % f
      \ifthenelse{\equal{\i}{6}}{\path[segstyle] 
        (${1-\sevensegShrink}*(P3)+\sevensegShrink*(P4)$) 
        -- ($\sevensegShrink*(P3)+{1-\sevensegShrink}*(P4)$);}{} % g
    }
  \end{tikzpicture}%
}

\newcommand{\sevensegnum}[2][]% sample characvters
{%                                          
  \ifthenelse{\equal{#2}{0}}{\sevenseg[#1]{{1,1,1,1,1,1,0,}}}{%
  \ifthenelse{\equal{#2}{1}}{\sevenseg[#1]{{0,1,1,0,0,0,0,}}}{%
  \ifthenelse{\equal{#2}{2}}{\sevenseg[#1]{{1,1,0,1,1,0,1,}}}{%
  \ifthenelse{\equal{#2}{3}}{\sevenseg[#1]{{1,1,1,1,0,0,1,}}}{%
  \ifthenelse{\equal{#2}{4}}{\sevenseg[#1]{{0,1,1,0,0,1,1,}}}{%
  \ifthenelse{\equal{#2}{5}}{\sevenseg[#1]{{1,0,1,1,0,1,1,}}}{%
  \ifthenelse{\equal{#2}{6}}{\sevenseg[#1]{{1,0,1,1,1,1,1,}}}{%
  \ifthenelse{\equal{#2}{7}}{\sevenseg[#1]{{1,1,1,0,0,0,0,}}}{%
  \ifthenelse{\equal{#2}{8}}{\sevenseg[#1]{{1,1,1,1,1,1,1,}}}{%
  \ifthenelse{\equal{#2}{9}}{\sevenseg[#1]{{1,1,1,1,0,1,1,}}}{%
  \ifthenelse{\equal{#2}{A}}{\sevenseg[#1]{{1,1,1,0,1,1,1,}}}{%
  \ifthenelse{\equal{#2}{B}}{\sevenseg[#1]{{0,0,1,1,1,1,1,}}}{%
  \ifthenelse{\equal{#2}{C}}{\sevenseg[#1]{{0,0,0,1,1,0,1,}}}{%
  \ifthenelse{\equal{#2}{D}}{\sevenseg[#1]{{0,1,1,1,1,0,1,}}}{%
  \ifthenelse{\equal{#2}{E}}{\sevenseg[#1]{{1,0,0,1,1,1,1,}}}{%
  \ifthenelse{\equal{#2}{F}}{\sevenseg[#1]{{1,0,0,0,1,1,1,}}}{%
  {\sevenseg[#1]{{0,0,0,0,0,0,0,}}}}}}}}}}}}}}}}}}}%
}


%%%%%%%%%%%%%%%%%%%%%%%%%%%%%%%%%%%%%%%%%%%%%%%%%%%%%%%%
\pagestyle{fancy}
\fancyhead[l]{Rüzgar Erik}
\fancyhead[c]{\mytitle}
\fancyhead[r]{\righthead}
\fancyfoot[c]{\thepage}
\renewcommand{\headrulewidth}{0.2pt}
\setlength{\headheight}{15pt}
%%%%%%%%%%%%%%%%%%%%%%%%%%%%%%%%%%%%%%%%%%%%%%%%%%%%%%%%
\usepackage{parskip}
\setlength{\parindent}{0pt} % Disable paragraph indentation
%%%%%%%%%%%%%%%%%%%%%%%%%%%%%%%%%%%%%%%%%%%%%%%%%%%%%%%%




\lstdefinestyle{vhdlstyle}{
    language=VHDL,
    backgroundcolor=\color{white},
    basicstyle=\ttfamily,
    keywordstyle=\color{blue},
    commentstyle=\color{green},
    stringstyle=\color{red},
    numbers=left,
    numberstyle=\tiny\color{gray},
    stepnumber=1,
    numbersep=5pt,
    tabsize=2,
    showspaces=false,
    showstringspaces=false,
}



\hypersetup{
  pdfauthor={Rüzgar Erik},
  pdftitle={\mytitle},
  pdfsubject={\righthead},
  pdfcreator={LaTeX},
  pdfproducer={Rüzgar Erik}
}






\begin{document}


\begin{titlepage}
    \begin{figure}[h] % Places the figure at the right
        \begin{flushright}
        \includegraphics[width=0.3\textwidth]{logo_laci.png} % Change to your image name
            
        \end{flushright}
        \hfill
    \end{figure}

    \centering
    \vspace*{1in}
    
    \Huge
    \textbf{Introduction to Logic Design} \\
    \textbf{EEF205E} \\

    \vspace{0.5in}

    \Large
    \textbf{\mytitle} \\
    
    \vspace{0.5in}

    \large
    \textbf{Rüzgar Erik} \\
    \textbf{040240783} \\

    \vspace{0.5in}
    
    \Large
    Istanbul Technical University \\
    Faculty of Electrical and Electronics Engineering \\
    
    \vfill


\end{titlepage}



\section*{Part 1}

\begin{enumerate}%first part questions
    \item  ABCD TO 7-SEGMENT DISPLAY Decoder Design With Minimum Number of Gates
    
    \textbf{Solution:}

    \begin{enumerate}
        \item Finding the truth table of the 7-segment display:

        \begin{center}
  

          \end{center}

          The truth table of the 7-segment display is as follows:

          \begin{table}[h!]
            \centering
            \rowcolors{2}{gray!10}{white}
            \setlength{\tabcolsep}{8pt} % Adjust column spacing
            \renewcommand{\arraystretch}{1.3} % Adjust row spacing
            \begin{tabular}{c|cccc|ccccccc}
            \rowcolor{cyan!20}
            \textbf{7 segment}&\textbf{A} & \textbf{B} & \textbf{C} & \textbf{D} & \textbf{a} & \textbf{b} & \textbf{c} & \textbf{d} & \textbf{e} & \textbf{f} & \textbf{g} \\ 
            \midrule
            \sevenseg[size = 1em]{{1,1,1,1,1,1,0}} & 0 & 0 & 0 & 0 & 1 & 1 & 1 & 1 & 1 & 1 & 0 \\ % 0
            \sevenseg[size = 1em]{{0,1,1,0,0,0,0}} & 0 & 0 & 0 & 1 & 0 & 1 & 1 & 0 & 0 & 0 & 0 \\ % 1
            \sevenseg[size = 1em]{{1,1,0,1,1,0,1}} & 0 & 0 & 1 & 0 & 1 & 1 & 0 & 1 & 1 & 0 & 1 \\ % 2
            \sevenseg[size = 1em]{{1,1,1,1,0,0,1}} & 0 & 0 & 1 & 1 & 1 & 1 & 1 & 1 & 0 & 0 & 1 \\ % 3
            \sevenseg[size = 1em]{{0,1,1,0,0,1,1}} & 0 & 1 & 0 & 0 & 0 & 1 & 1 & 0 & 0 & 1 & 1 \\ % 4
            \sevenseg[size = 1em]{{1,0,1,1,0,1,1}} & 0 & 1 & 0 & 1 & 1 & 0 & 1 & 1 & 0 & 1 & 1 \\ % 5
            \sevenseg[size = 1em]{{1,0,1,1,1,1,1}} & 0 & 1 & 1 & 0 & 1 & 0 & 1 & 1 & 1 & 1 & 1 \\ % 6
            \sevenseg[size = 1em]{{1,1,1,0,0,0,0}} & 0 & 1 & 1 & 1 & 1 & 1 & 1 & 0 & 0 & 0 & 0 \\ % 7
            \sevenseg[size = 1em]{{1,1,1,1,1,1,1}} & 1 & 0 & 0 & 0 & 1 & 1 & 1 & 1 & 1 & 1 & 1 \\ % 8
            \sevenseg[size = 1em]{{1,1,1,1,0,1,1}} & 1 & 0 & 0 & 1 & 1 & 1 & 1 & 1 & 0 & 1 & 1 \\ % 9
            \end{tabular}
            \caption{7-Segment Display Truth Table for BCD Inputs}
            \label{tab:seven_segment_bcd}
            \end{table}
            
        The remaining 6 rows are not shown in the table since they are not used in the BCD representation. They will be placed as "don't care" in the Karnaugh map.


        \item Constructing the Karnaugh map for each segment:
        

            \begin{figure}[H]
                \centering
                \begin{karnaugh-map}[4][4][1][\(D\)][\(C\)][\(B\)][\(A\)]
                    \minterms{0,2,3,5,6,7,8,9}
                    \maxterms{1,4}
                    \autoterms[X]
                    \implicant{12}{10}
                    \implicantcorner
                    \implicant{3}{10}
                    \implicant{5}{15}
                \end{karnaugh-map}
                \caption{Kmap for a output}
            \end{figure}
            
            The optimized expression for \(a\) is: \(B'D' + BD + C + A\)



            \begin{figure}[H]
                \centering
                \begin{karnaugh-map}[4][4][1][\(D\)][\(C\)][\(B\)][\(A\)]
                    \minterms{0,1,2,3,4,7,8,9}
                    \maxterms{5,6}
                    \autoterms[X]
                    \implicantedge{0}{2}{8}{10}
                    \implicant{0}{8}
                    \implicant{3}{11}
                \end{karnaugh-map}
                \caption{Kmap for b output}
            \end{figure}
            
            The optimized expression for \(b\) is: \(B' + C'D' + CD\)


            \begin{figure}[H]
                \centering
                \begin{karnaugh-map}[4][4][1][\(D\)][\(C\)][\(B\)][\(A\)]
                    \minterms{0,1,3,4,5,6,7,8,9}
                    \maxterms{2}
                    \autoterms[X]
                    \implicant{0}{9}
                    \implicant{1}{11}
                    \implicant{4}{14}
                \end{karnaugh-map}
                \caption{Kmap for c output}
            \end{figure}
            
            The optimized expression for \(c\) is: \(C' + B + D\)


            \begin{figure}[H]
                \centering
                \begin{karnaugh-map}[4][4][1][\(D\)][\(C\)][\(B\)][\(A\)]
                    \minterms{0,2,3,5,6,8,9}
                    \maxterms{1,4,7}
                    \autoterms[X]
                    \implicantcorner
                    \implicant{12}{10}
                    \implicant{5}{13}
                    \implicant{2}{10}
                    \implicantedge{3}{2}{11}{10}

                \end{karnaugh-map}
                \caption{Kmap for d output}
            \end{figure}
            
            The optimized expression for \(d\) is: \(B'D'+B'C+BC'D+CD'+A\)


            \begin{figure}[H]
                \centering
                \begin{karnaugh-map}[4][4][1][\(D\)][\(C\)][\(B\)][\(A\)]
                    \minterms{0,2,6,8}
                    \maxterms{1,3,4,5,7,9}
                    \autoterms[X]
                    \implicantcorner
                    \implicant{2}{10}


                \end{karnaugh-map}
                \caption{Kmap for e output}
            \end{figure}
            
            The optimized expression for \(e\) is: \(B'D'+CD'\)
            



            \begin{figure}[H]
                \centering
                \begin{karnaugh-map}[4][4][1][\(D\)][\(C\)][\(B\)][\(A\)]
                    \minterms{0,4,5,6,8,9}
                    \maxterms{1,2,3,7}
                    \autoterms[X]
                    \implicant{0}{8}
                    \implicant{12}{10}
                    \implicantedge{4}{12}{6}{14}
                    \implicant{4}{13}


                \end{karnaugh-map}
                \caption{Kmap for f output}
            \end{figure}
            
            The optimized expression for \(f\) is: \(A + BC'+ BD' + C'D'\)


            \begin{figure}[H]
                \centering
                \begin{karnaugh-map}[4][4][1][\(D\)][\(C\)][\(B\)][\(A\)]
                    \minterms{2,3,4,5,6,8,9}
                    \maxterms{0,1,7}                    
                    \autoterms[X]
                    \implicant{2}{10}
                    \implicant{4}{13}
                    \implicant{12}{10}
                    \implicantedge{3}{2}{11}{10}

                \end{karnaugh-map}
                \caption{Kmap for g output}
            \end{figure}
            
            The optimized expression for \(g\) is: \(A + BC'+B'C+ CD'\)

        \newpage

        \item Design a four-bit combinational circuit 2’s complementor. (The output generates the 2’s complement of
        the input binary number.) Show that the circuit can be constructed with exclusive-OR gates. Can you
        predict what the output functions are for a five-bit 2’s complementor?

        \textbf{Solution:}

        Taking the 2's complement of a binary number is equivalent to inverting all the bits and adding 1 to the result. The circuit can be constructed with exclusive-OR gates as follows:

        \begin{table}[H]
            \centering
            \begin{tabular}{|c|c|c|c|c||c|c|c|c|c|}
            \hline
            \rowcolor{cyan!70}
            \multicolumn{5}{|c||}{Input} & \multicolumn{5}{c|}{Output (2's Complement)} \\
            \rowcolor{cyan!20}
            \hline
            a & b & c & d & Binary & x & y & z & t & Binary \\
            \hline
            0 & 0 & 0 & 0 & 0000 & 0 & 0 & 0 & 0 & 0000 \\
            0 & 0 & 0 & 1 & 0001 & 1 & 1 & 1 & 1 & 1111 \\
            0 & 0 & 1 & 0 & 0010 & 1 & 1 & 1 & 0 & 1110 \\
            0 & 0 & 1 & 1 & 0011 & 1 & 1 & 0 & 1 & 1101 \\
            0 & 1 & 0 & 0 & 0100 & 1 & 1 & 0 & 0 & 1100 \\
            0 & 1 & 0 & 1 & 0101 & 1 & 0 & 1 & 1 & 1011 \\
            0 & 1 & 1 & 0 & 0110 & 1 & 0 & 1 & 0 & 1010 \\
            0 & 1 & 1 & 1 & 0111 & 1 & 0 & 0 & 1 & 1001 \\
            1 & 0 & 0 & 0 & 1000 & 1 & 0 & 0 & 0 & 1000 \\
            1 & 0 & 0 & 1 & 1001 & 0 & 1 & 1 & 1 & 0111 \\
            1 & 0 & 1 & 0 & 1010 & 0 & 1 & 1 & 0 & 0110 \\
            1 & 0 & 1 & 1 & 1011 & 0 & 1 & 0 & 1 & 0101 \\
            1 & 1 & 0 & 0 & 1100 & 0 & 1 & 0 & 0 & 0100 \\
            1 & 1 & 0 & 1 & 1101 & 0 & 0 & 1 & 1 & 0011 \\
            1 & 1 & 1 & 0 & 1110 & 0 & 0 & 1 & 0 & 0010 \\
            1 & 1 & 1 & 1 & 1111 & 0 & 0 & 0 & 1 & 0001 \\
            \hline
            \end{tabular}
            \caption{4-bit 2's Complement Truth Table}
            \label{tab:twos_complement}
            \end{table}
    
            
            Generating KMamps for each output bit:

                \begin{figure}[H]
                    \centering
                    \begin{karnaugh-map}[4][4][1][\(D\)][\(C\)][\(B\)][\(A\)]
                        \minterms{1,2,3,4,5,6,7,8}
                        \maxterms{0,9,10,11,12,13,14,15}
                        \implicant{8}{8}
                        \implicant{4}{6}
                        \implicant{1}{7}
                        \implicant{3}{6}
                    \end{karnaugh-map}
                \caption{Kmap for x output}
                \end{figure}

                The equation for \(x\) is: \(a'b+ a'd + a'c + ab'c'd'\) for converting this to XOR gates, we can use the following equation: 

                \begin{align}
                    x &= a'b+ a'd + a'c + ab'c'd' \\
                    x &= a' \left( b + c + d \right) + ab'c'd' \\
                    x &= a' \left( b + d + c \right) + a \left( b + c + d \right)'\\
                    x &= a \oplus \left( b + c + d \right)
                \end{align}


                \begin{figure}[H]
                    \centering
                    \begin{karnaugh-map}[4][4][1][\(D\)][\(C\)][\(B\)][\(A\)]
                        \minterms{1,2,3,4,9,10,11,12}
                        \maxterms{0,5,6,7,8,13,14,15}
                        \implicantedge{1}{3}{9}{11}
                        \implicantedge{3}{2}{11}{10}
                        \implicant{4}{12}
                    \end{karnaugh-map}
                \caption{Kmap for y output}
                \end{figure}
                
                The equation for \(y\) is \(bc'd' + b'd + b'c\) for converting this to XOR gates, we can use the following equation:

                \begin{align}
                    y &= bc'd' + b'd + b'c \\
                    y &= b \left( c'd' + d + c \right) \\
                    y &= b \oplus \left( c + d \right)
                \end{align}

                \begin{figure}[H]
                    \centering
                    \begin{karnaugh-map}[4][4][1][\(D\)][\(C\)][\(B\)][\(A\)]
                        \minterms{1,2,5,6,9,10,13,14}
                        \maxterms{0,3,4,7,8,11,12,15}
                        \implicant{2}{10}
                        \implicant{1}{9}
                    \end{karnaugh-map}
                \caption{Kmap for z output}
            \end{figure}

            The equation for \(z\) is \(c'd + cd'\) for converting this to XOR gates, we can use the following equation:

            \begin{align}
                z &= c'd + cd' \\
                z &= c \oplus d
            \end{align}

                \begin{figure}[H]
                    \centering
                    \begin{karnaugh-map}[4][4][1][\(D\)][\(C\)][\(B\)][\(A\)]
                        \minterms{1,3,5,7,9,11,13,15}
                        \maxterms{0,2,4,6,8,10,12,14}
                        \implicant{1}{11}
                    \end{karnaugh-map}
                \caption{Kmap for t output}
                \end{figure}

                The equation for \(t\) is \(d\) .
            
        The final conversion functions are:

        \begin{align}
            x &= a \oplus \left( b + c + d \right) \\
            y &= b \oplus \left( c + d \right) \\
            z &= c \oplus d \\
            t &= d
        \end{align}


        We can expect that for a 5 bit 2's complementor the inputs are \(A_4, A_3, A_2, A_1, A_0\) and the outputs are \(X_4, X_3, X_2, X_1, X_0\). The equations for the outputs can be derived by following the same steps as above. The final equations will be:


        \begin{align}
            X_4 &= A_4 \oplus \left( A_3 + A_2 + A_1 + A_0 \right) \\
            X_3 &= A_3 \oplus \left( A_2 + A_1 + A_0 \right) \\
            X_2 &= A_2 \oplus \left( A_1 + A_0 \right) \\
            X_1 &= A_1 \oplus A_0 \\
            X_0 &= A_0
        \end{align}

    \end{enumerate}



\end{enumerate}%first part questions





\end{document}